\textbf{SYNTAX DER DEUTSCHEN GEGENWARTSSPRACHE}

\emph{\textbf{SS 2000/2001}}

\textbf{Satztopologie des Gegenwartsdeutschen im Kontrast}

Die finite Verbform und die infinite(n) Verbform(en) zerlegen
\textbf{deutsche Sätze in drei Stellungsfelder}: in das Vorfeld,
Mittelfeld und Nachfeld (Tabelle 1).

\begin{longtable}[]{@{}
  >{\raggedright\arraybackslash}p{(\columnwidth - 10\tabcolsep) * \real{0.18}}
  >{\raggedright\arraybackslash}p{(\columnwidth - 10\tabcolsep) * \real{0.09}}
  >{\raggedright\arraybackslash}p{(\columnwidth - 10\tabcolsep) * \real{0.12}}
  >{\raggedright\arraybackslash}p{(\columnwidth - 10\tabcolsep) * \real{0.31}}
  >{\raggedright\arraybackslash}p{(\columnwidth - 10\tabcolsep) * \real{0.14}}
  >{\raggedright\arraybackslash}p{(\columnwidth - 10\tabcolsep) * \real{0.17}}@{}}
\toprule
\multicolumn{6}{l}{\begin{minipage}[b]{\linewidth}\raggedright
\textbf{TABELLE 1: STELLUNGSFELDER IN DEUTSCHEN SÄTZEN}
\end{minipage}} \\
\midrule
\endhead
\textbf{VORFELD} & \textbf{LSK}

\textbf{C\textsuperscript{0}} & \textbf{KLITIKA} & \textbf{MITTELFELD} &
\textbf{RSK}

\textbf{V\textsuperscript{0}} & \textbf{NACHFELD} \\
(1) Der Arbeiter & \textbf{hat} & & dem Vorarbeiter den Schlüssel &
gegeben. & \\
(2) & \textbf{Hat} & & der Arbeiter dem Vorarbeiter den Schlüssel &
gegeben? & \\
(3) Wer & \textbf{hat} & & dem Vorarbeiter den Schlüssel & gegeben? & \\
(4) Was & \textbf{hat'} & s & dir eigentlich & genutzt, & dass du
Widerstand geleistet hast? \\
(5) {[}Ich glaube,{]} & \emph{\textbf{dass}} & & all deine Versuche
sinnlos & \textbf{sind}. & \\
(6) {[}Ich frage mich,{]} & \emph{\textbf{ob}} & & all deine Versuche
sinnlos & \textbf{sind}. & \\
(7) {[}Er weiß nicht,{]} \emph{\textbf{was}} & {[}+C{]} & & ihre Worte &
bedeuten \textbf{sollen}. & \\
(8) {[}Sie fragte mich,{]} \emph{\textbf{wann}} & {[}+C{]} & & sie &
kommen \textbf{sollte}. & \\
(9) {[}Der Mann,{]} \emph{\textbf{der}} & {[}+C{]} & & sich Wyatt Earp &
\textbf{nannte}, & \\
(10) {[}Er führte eine Behauptung an,{]} \textbf{die} & {[}+C{]} & &
trotz aller überzeugender Nachweise & falsch \textbf{war}. & \\
\multicolumn{6}{l}{SYMBOLE: LSK (Linke Satzklammer) oder
C\textsuperscript{0} Position des Komplementierers (angl.
complementizer), V\textsuperscript{0} zugrunde liegende Verbposition
oder RSK (Rechte Satzklammer)} \\
\bottomrule
\end{longtable}

\begin{longtable}[]{@{}
  >{\raggedright\arraybackslash}p{(\columnwidth - 14\tabcolsep) * \real{0.14}}
  >{\raggedright\arraybackslash}p{(\columnwidth - 14\tabcolsep) * \real{0.11}}
  >{\raggedright\arraybackslash}p{(\columnwidth - 14\tabcolsep) * \real{0.09}}
  >{\raggedright\arraybackslash}p{(\columnwidth - 14\tabcolsep) * \real{0.09}}
  >{\raggedright\arraybackslash}p{(\columnwidth - 14\tabcolsep) * \real{0.15}}
  >{\raggedright\arraybackslash}p{(\columnwidth - 14\tabcolsep) * \real{0.11}}
  >{\raggedright\arraybackslash}p{(\columnwidth - 14\tabcolsep) * \real{0.12}}
  >{\raggedright\arraybackslash}p{(\columnwidth - 14\tabcolsep) * \real{0.17}}@{}}
\toprule
\endhead
\multicolumn{8}{l}{\textbf{Tabelle 3: Vorvorfeld, Nachnachfeld,
Klitische Reihung (Deutsch -- Slowenisch)}} \\
& \textbf{Satz (=CP)} & & & & & & \\
\textbf{Vorvorfeld} & \textbf{Vorfeld} & \textbf{LSK}

\textbf{C\textsuperscript{0}} & \textbf{Klitika} & \textbf{Mittelfeld} &
\textbf{RSK}

\textbf{V\textsuperscript{0}} & \textbf{Nachfeld} &
\textbf{Nachnachfeld} \\
\textbf{Deutsch:} & & & & & & & \\
(1) und & dann & flog & & der Vogel & davon. & & \\
(2) oder & er & glaubt & & es mir nicht. & & & \\
(3) Oder & & glaubt & & sie' s nicht? & & & \\
(4) Aber & wer & glaubt & & mir schon? & & & \\
(5) denn & morgen & rufen & & sie uns schon & an. & & \\
(6) Ja, & ich & hab' & s & natürlich & geahnt. & & \\
(7) Vielleicht, & mehr & weiß & & ich nicht. & & & \\
(8) Leider, & so & ist & & das eben. & & & \\
(9) Tja, & leider & ist & & das eben so. & & & \\
(10) Mein Gott, & trotzdem & läuft & & er mir immer & nach. & & \\
(11) Nun, & sie & glaubt & & mir nicht, & & & oder? \\
(12) Kurzum: & Sie & rufen & & mich dann & an, & & ja? \\
(13) & Sie & kennen & & ihn doch gut, & & & nicht wahr? \\
(14) & Sie & wissen' & s & doch schon, & & & nicht? \\
(15) So, & nun & hör & & mir mal gut & zu, & & mein lieber. \\
\textbf{Slowenisch} & & & & & & & \\
\textbf{Vorvorfeld} & \textbf{Vorfeld} & \textbf{LSK}

\textbf{C\textsuperscript{0}} & \textbf{Klitika} & \textbf{Mittelfeld} &
\textbf{RSK}

\textbf{V\textsuperscript{0}} & \textbf{Nachfeld} &
\textbf{Nachnachfeld} \\
\textbf{(7)} in & & da & te je & Igor sinoči & povabil & na večerjo & \\
\textbf{(8)} toda & Igor & & te je & sinoči & povabil & na večerjo & \\
(1) in & sinoči & & te je & Igor & iskal, & & a veš? \\
(2) toda & Janez & & se bo & vsekakor & pritožil, & & ne? \\
(3) Žal, & takšne & & so & pač navade, & & & moj dragi. \\
(4) Da / ja, & seveda & & sem ga & takoj & opazil. & & \\
(5) & vendar & & sem jo & kmalu & pozabil, & & veš? \\
(6) Skratka: & zvečer & & me & & pokličite, & & prav? \\
\bottomrule
\end{longtable}

\textbf{Anmerkungen zu Tabelle 1:}

\begin{itemize}
\item
  \textbf{Die finite Verbform in deutschen Hauptsätzen} besetzt die
  \textbf{zweite syntaktische Position} (die man mit
  C\textsuperscript{0}, d.h. der Kopf-Position in der
  Komplementiererphrase CP im Rahmen der Rektions-und Bindungstheorie,
  gleichsetzen kann).
\end{itemize}

\begin{itemize}
\item
  \textbf{Hauptsätze}: Während das Vorfeld in Aussagesätzen und
  Ergänzungsfragesätzen von einer Phrase gefüllt wird, bleibt das
  \textbf{Vorfeld in Entscheidungsfragesätzen} (Ja-/Nein-Fragen)
  unbesetzt (weil das finite Verb die erste sichtbare oder hörbare
  Wortform ist, spricht man oft von \textbf{V-1-Stellung}). Auch
  \textbf{Imperativsätze} und \textbf{uneingeleitete Wunschsätze}
  gehören zu den Sätzen mit V-1-Stellung, zuweilen auch Exklamativsätze
  ("Ausrufesätze").
\end{itemize}

\begin{itemize}
\item
  \textbf{In Nebensätzen} (abhängigen Sätzen mit einem Subjunktor,
  Interrogativpronomen, Interrogativadverb oder Relativpronomen als
  Einleitungselement) besetzt die finite Verbform hingegen den Knoten
  V\textsuperscript{0}, d.h. die angenommene zugrunde liegende
  Verbposition in der Verbalphrase (weil das finite Verb in Nebensätzen
  oft als letzte hörbare oder sichtbare Wortform auftritt, spricht man
  oberflächensyntaktisch oft von \textbf{V-Letzt-Stellung}).
\end{itemize}

\begin{itemize}
\item
  \textbf{Einleitungselemente} von Nebensätzen \textbf{verhindern die
  Zweitstellung} der finiten Verbform. Gemäß der Rektions- und
  Bindungstheorie stehen \textbf{Subjunktoren} wie z. B. \emph{dass},
  \emph{ob, nachdem, wenn} (Einleitungselemente, aber keine Satzglieder)
  selbst im Knoten C\textsuperscript{0} und lassen daher kein finites
  Verb zu. \textbf{W-/d-Phrasen} wie z. B. \emph{was, wann, warum, was
  für ein; der, die, das} (gleichzeitig Einleitungselemente und
  Satzglieder) sind maximale Phrasen (sie beziehen sich auf
  Nominalphrasen oder Präpositionalphrasen und können eine solche
  Struktur aufweisen) und treten daher im Vorfeld des Satzes auf, d.h.
  in der Spezifizierposition der Komplementiererphrase. Sie stellen eine
  syntaktische Beziehung zum Kopf der Komplementiererphrase
  C\textsuperscript{0} her, die die Zweitstellung des finiten Verbs
  verhindert, wenn der Satz mit w-/d-Phrase als Einleitungselement ein
  abhängiger Satz ist (eine Lösung im Rahmen der Rektions- und
  Bindungstheorie: der Knoten C\textsuperscript{0} ist nicht wirklich
  leer, sondern wird aufgrund einer speziellen Art von Kongruenz
  zwischen der Position C\textsuperscript{0} und der w-/d-Phrase, die in
  der Spezifiziererposition im Vorfeld steht, von einem nicht hör- oder
  sichtbaren Merkmal {[}+C{]} freigehalten, der zum lexikalischen
  Eintrag des Einleitungselementes gehört und Verb-Letzt-Stellung
  auslöst oder nicht; Subjunktoren haben den lexikalischen Eintrag
  {[}+C{]}, w-/d-Phrasen haben den lexikalischen Eintrag {[}+/-C{]},
  d.h. wenn ein Satz mit w-/d-Phrase als Einleitungselement in einen
  anderen Satz eingebettet und von einem Verb im Obersatz regiert wird,
  weiß ein deutscher Muttersprachler, dass der abhängige Satz mit
  Verb-Letzt-Stellung realisierbar ist).
\end{itemize}

\begin{itemize}
\item
  Die finiten Verbformen können \textbf{je nach Satztyp} (Hauptsatz oder
  Nebensatz) entweder nur an einer \textbf{syntaktischen Position} oder
  sogar an zwei verschiedenen syntaktischen Positionen auftreten können:
  in Hauptsätzen treten die Teile der Verbalphrasen
  \textbf{diskontinuierlich} auf (eine im Sprachenvergleich markiertere
  Option), in Nebensätzen hingegen in \textbf{Kontaktposition} (eine im
  Sprachenvergleich weniger markierte Möglichkeit). Hauptsätze zeigen
  eine \textbf{komplexere Satzstruktur} als die von der Funktion her
  spezialisierteren Nebensätze.
\item
  Das \textbf{Nachfeld} eines deutschen Satzes steht im Gegensatz zu
  Vor- und Mittelfeld meist leer und kann lediglich nach besonderen
  Regularitäten von bestimmten (in der Regel formal komplexeren)
  Satzkonstituenten besetzt werden (in der germanistischen Literatur
  spricht man oft von \textbf{Ausklammerung oder Ausrahmung}).
\end{itemize}

(vgl. auch Abraham 1995: Deutsche Sprache im Sprachenvergleich,
Tübingen)

\textbf{Anmerkungen zur Tabelle 3:}

\begin{itemize}
\item
  Vorvorfeld und Nachnachfeld \textbf{gehören nicht zum Satz}(verband)
  selbst, sondern sind locker damit verbunden: eigener Intonationsbogen,
  Pause nach Vorvorfeldelement möglich, in der Interpunktion durch
  Doppelpunkt oder Komma gekennzeichnet; keine Satzglieder im Satz,
  sondern eher Satzäquivalente. Auch nach den koordinierenden
  Konjunktionen kann eine Pause auftreten (die in der Schrift gewöhnlich
  durch einen Doppelpunkt markiert wird).
\item
  Als \textbf{Elemente im Vorvorfeld deutscher Sätze} erscheinen:
  koordinierende Konjunktionen (Junktoren), Interjektionen,
  Satzadverbien (Modalwörter) bzw. Satzadverbiale, Gliederungspartikeln
  (Antwortpartikeln). Entsprechendes gilt für slowenische Sätze.
\item
  Als \textbf{Elemente im Nachnachfeld deutscher Sätze} erscheinen:
  Gliederungspartikeln, NPs (Adressat), Negationspartikeln u.a.
  Entsprechendes gilt für slowenische Sätze.
\end{itemize}

\begin{longtable}[]{@{}
  >{\raggedright\arraybackslash}p{(\columnwidth - 16\tabcolsep) * \real{0.15}}
  >{\raggedright\arraybackslash}p{(\columnwidth - 16\tabcolsep) * \real{0.07}}
  >{\raggedright\arraybackslash}p{(\columnwidth - 16\tabcolsep) * \real{0.11}}
  >{\raggedright\arraybackslash}p{(\columnwidth - 16\tabcolsep) * \real{0.17}}
  >{\raggedright\arraybackslash}p{(\columnwidth - 16\tabcolsep) * \real{0.07}}
  >{\raggedright\arraybackslash}p{(\columnwidth - 16\tabcolsep) * \real{0.10}}
  >{\raggedright\arraybackslash}p{(\columnwidth - 16\tabcolsep) * \real{0.10}}
  >{\raggedright\arraybackslash}p{(\columnwidth - 16\tabcolsep) * \real{0.07}}
  >{\raggedright\arraybackslash}p{(\columnwidth - 16\tabcolsep) * \real{0.15}}@{}}
\toprule
\multicolumn{9}{l}{\begin{minipage}[b]{\linewidth}\raggedright
\textbf{Tabelle 2: STELLUNGSFELDER IN SLOWENISCHEN SÄTZEN}
\end{minipage}} \\
\begin{minipage}[b]{\linewidth}\raggedright
\textbf{VORFELD}
\end{minipage} & \begin{minipage}[b]{\linewidth}\raggedright
\end{minipage} &
\multicolumn{2}{l}{\begin{minipage}[b]{\linewidth}\raggedright
\textbf{LINKES MITTELFELD}
\end{minipage}} & \begin{minipage}[b]{\linewidth}\raggedright
\end{minipage} &
\multicolumn{2}{l}{\begin{minipage}[b]{\linewidth}\raggedright
\textbf{RECHTES MITTELFELD}
\end{minipage}} & \begin{minipage}[b]{\linewidth}\raggedright
\end{minipage} & \begin{minipage}[b]{\linewidth}\raggedright
\textbf{NACHFELD}
\end{minipage} \\
\multicolumn{2}{l}{\begin{minipage}[b]{\linewidth}\raggedright
C-SYSTEM (CP)
\end{minipage}} & \begin{minipage}[b]{\linewidth}\raggedright
\end{minipage} &
\multicolumn{2}{l}{\begin{minipage}[b]{\linewidth}\raggedright
I-SYSTEM (IP)
\end{minipage}} &
\multicolumn{4}{l}{\begin{minipage}[b]{\linewidth}\raggedright
V-SYSTEM (VP)
\end{minipage}} \\
\begin{minipage}[b]{\linewidth}\raggedright
\textbf{CP:}

\textbf{XP}
\end{minipage} & \begin{minipage}[b]{\linewidth}\raggedright
\textbf{C\textsuperscript{1}:}

\textbf{C\textsuperscript{0} + CL}
\end{minipage} & \begin{minipage}[b]{\linewidth}\raggedright
\textbf{TOPP:}

\textbf{XP}
\end{minipage} & \begin{minipage}[b]{\linewidth}\raggedright
\textbf{IP:}

\textbf{NP}
\end{minipage} & \begin{minipage}[b]{\linewidth}\raggedright
\textbf{I\textsuperscript{1}:}

\textbf{I\textsuperscript{0}}
\end{minipage} & \begin{minipage}[b]{\linewidth}\raggedright
\textbf{VP:}

\textbf{NP}
\end{minipage} & \begin{minipage}[b]{\linewidth}\raggedright
\textbf{V\textsuperscript{1}:}

\textbf{PP}
\end{minipage} & \begin{minipage}[b]{\linewidth}\raggedright
\textbf{V\textsuperscript{1}:}

\textbf{V\textsuperscript{0}}
\end{minipage} & \begin{minipage}[b]{\linewidth}\raggedright
\textbf{V\textsuperscript{1}:}

\textbf{NP}
\end{minipage} \\
\midrule
\endhead
(1) & & & Zdrav način življenja \textsubscript{j} & pač ne bi mogel
\textsubscript{i} & j & ob zadovoljivo pestri hrani & sprožiti & prav
takega teka, ki ustreza potrebam \ldots{} \\
(2) & da & & zdrav način življenja \textsubscript{j} & pač ne bi mogel
\textsubscript{i} & j & ob z. p. hrani & sprožiti & prav takega teka, ki
ustreza potrebam... \\
(3) & da & ob zadostno pestri hrani \textsubscript{k} & zdrav način
življenja \textsubscript{j} & ni mogel \textsubscript{i} & j & k

pri vsakomur & sprožiti & prav takega teka, ki ustreza potrebam... \\
(4) Ob zad. pest. hrani \textsubscript{k} & bi & na vsak način
\textsubscript{l} & zdrav način življenja \textsubscript{j} & utegnil
\textsubscript{i} & j & k l

pri vsakomur & sprožiti & prav tak tek, ki ustreza potrebam ... \\
(5) & & Ob zadostno pestri hrani & zdrav način življenja
\textsubscript{j} & sproži \textsubscript{i} & j & k & i & prav tak tek,
ki ustreza potrebam... \\
(6) Zdrav način življ. \textsubscript{j} & & ob zadostno pestri hrani
\textsubscript{k} & j & sproži \textsubscript{i} & j & k & i & prav tak
tek, ki ustreza potrebam... \\
(7) & Ali bo & & zdrav način življenja \textsubscript{j} & i & j & ob z.
p. hrani & sprožil & prav tak tek? \\
(8) & & & & Sproži \textsubscript{i} & zdrav način ž. & ob z. p. hrani &
i & prav tak tek? \\
(9) V katerem primeru \textsubscript{k} & bo & & zdrav način življenja
\textsubscript{j} & i & j & k & sprožil & prav tak tek? \\
(10) Kaj \textsubscript{j} & bi & na vsak način \textsubscript{l} & j &
utegnilo \textsubscript{i} & j & ob z. p. hrani

l & sprožiti & prav tak tek? \\
(11) Komu \textsubscript{n} & da naj & na vsak način \textsubscript{l} &
\emph{pro} \textsubscript{j} & pišem \textsubscript{i} ? & j & n & i
& \\
(12) \ldots, ki \textsubscript{j} & bi ga \textsubscript{m} & na vsak
način \textsubscript{l} & j & utegnil \textsubscript{i} & j

l & ob z. p. hrani & sprožiti & m

pri vsakomur. \\
(13) & Vrni se & & \emph{pro} \textsubscript{j} & i & j & čez nekaj dni
& i & v domovino. \\
(14) Čez nekaj dni \textsubscript{k} & se & & \emph{pro}
\textsubscript{j} & vrni \textsubscript{i} & j & k & i & v domovino. \\
(15) Čez nekaj dni \textsubscript{k} & se & & še TI \textsubscript{j} &
vrni \textsubscript{i} & j & k & i & v domovino. \\
\multicolumn{9}{l}{\textbf{Legende: CP} (engl. complementizer phrase;
dt. Komplementiererphrase); \textbf{CL} (engl. clitics,;dt. Klitika);
\textbf{TOPP} (engl. topic phrase, dt. Topikphrase); \textbf{IP} (engl.
inflectional phrase, Flexionsphrase), gliederbar in \textbf{NegP, AGRP,
TP} (engl. negation phrase, agreement phrase, tense phrase; dt.
Negationsphrase oder Positionsphrase), Kongruenzphrase, Tempusphrase);
\textbf{VP} (engl. verb phrase; dt. Verbalphrase); \textbf{NP} (engl.
noun phrase; dt. Nominalphrase); \textbf{PP} (engl. prepositional
phrase; dt. Präpositionalphrase); \textbf{XP} (irgeind eine Phrase, d.h.
eine Maximalprojektion); \textbf{V\textsuperscript{0}} (zugrunde
liegende Verbposition); \textbf{V\textsuperscript{1}} (Ebene zwischen
Maximalprojektion und lexikalischer Ebene); Koindizierung: i, j, k, l,
m, n (\emph{i} für Verbformen, \emph{j} für die Subjekts-NP, \emph{k}
und \emph{l} für PP, \emph{m} für NP als Akkusativobjekt, \emph{n} für
NP als Dativobjekt); \emph{pro} (nichtrealisierte pronominalisierte
Subjekts-NP).} \\
\bottomrule
\end{longtable}

\textbf{Anmerkungen zur Tabelle 2:}

Nach den Vorgaben der Rektions- und Bindungstheorie (vgl. Cook \& Newson
\textsuperscript{2}1996 und Haftka 1996) ergibt sich ein
differenzierteres Bild der Stellungsfelder in slowenischen Sätzen
(Tabelle 2), deren Besonderheiten man im Vergleich zu den
Stellungsfeldern in deutschen Sätzen folgendermaßen beschreiben kann:

\begin{itemize}
\item
  Die syntaktische \textbf{Position eine slowenischen Verbs} im Satz
  scheint \textbf{von seiner Form abhängig} zu sein: die klitisierten
  Formen des Auxiliarverbs \emph{biti} (dt. "sein") erscheinen in der
  klitischen Reihung unmittelbar nach der Position C\textsuperscript{0},
  Vollverben vor allem in den tieferen Positionen V\textsuperscript{0}
  und I\textsuperscript{0}, die Imperativformen möglicherweise auch vor
  der klitischen Reihung im Knoten C\textsuperscript{0} (d.h. dort, wo
  deutsche finite Verbformen erscheinen).
\item
  Ein slowenischer Aussagesatz wird durch die klitische
  Reihung\footnote{In der klitischen Reihung in einem slowenischen Satz
    erscheinen die klitisierten Formen des Auxiliarverbs \emph{biti}
    (dt. "sein"), das klitisierte Reflexivpronomen \emph{se} (dt.
    "sich"), die klitisierten Personalpronomina im Genitiv, Dativ und
    Akkusativ (z.B. \emph{mu, ga, jim}, dt. "ihm", "ihn", "ihnen") und
    unflektierte Lexeme (z.B. \emph{bi}, dt. "würde"; \emph{naj}, dt.
    "soll").} sowie durch Verbformen in I\textsuperscript{0} und
  V\textsuperscript{0} in \textbf{vier Stellungsfelder} geteilt, die wir
  als Vorfeld, linkes (thematisches) Mittelfeld, rechtes (rhematisches)
  Mittelfeld und Nachfeld bezeichnen. Während ein deutscher Aussagesatz
  immer ein langes Mittelfeld aufweist (d.h. nicht durch eine finite
  Verbform zerlegt wird), kann ein slowenischer über ein langes
  Mittelfeld (bei diskontinuierlicher Position der Verbformen in der
  klitischen Reihung und in V\textsuperscript{0}) oder ein kurzes
  Mittelfeld (bei diskontinuierlicher Position der Verbfomen in
  I\textsuperscript{0} und V\textsuperscript{0}) verfügen.
\item
  Während in deutschen Sätzen das Nachfeld oft leer ist, kommt es in
  slowenischen Sätzen häufiger vor, dass das (linke oder rechte)
  \textbf{Mittelfeld leer} ist und \textbf{stattdessen das Nachfeld} mit
  einem oder mehreren rhematischen Satzgliedern \textbf{besetzt} ist.
\item
  \textbf{Negationswörter} wie \emph{ne} und seine flektierten Fomen
  sowie \textbf{Partikeln mit abtönender Bedeutung} (z.B. \emph{pač,}
  dt. "halt", "eben", "doch") bilden die Grenze zwischen dem linken
  (thematischen) und dem rechten (rhematischen) Mittelfeld und somit
  auch die Grenze zwischen Thema und Rhema. Entsprechendes gilt auch für
  die deutsche Negationspartikel \emph{nicht} und die deutschen
  Abtönungspartikeln (z.B. \emph{halt, eben, doch, ja}).
\item
  Das im heutigen Standardslowenischen \textbf{unflektierte Lexem}
  \emph{bi} (dt. "würde") scheint in negierten Sätzen sowohl in der
  klitischen Reihung unmittelbar nach dem Knoten C\textsuperscript{0}
  als auch als Klitikon vor der syntaktischen Position
  I\textsuperscript{0} auftreten zu können, und zwar unmittelbar vor der
  Satznegation \emph{ne} (dt. "nicht"). In affirmativen slowenischen
  Sätzen erscheint das unflektierte Lexem \emph{bi} in der klitischen
  Reihung unmittelbar nach dem Knoten C\textsuperscript{0}.
\end{itemize}
